
% preamble --------------------------
\documentclass{article}

% Packages
\usepackage{algorithm}
\usepackage{algpseudocode}
\usepackage{amsmath,amsthm,enumitem}
\usepackage{amssymb}
\usepackage{graphicx}
\usepackage{xcolor}
\usepackage[a4paper, total={6in, 8in}, margin=1.0in]{geometry}
% Information
\title{PA2 - Alternating-Bit Protocol}
\date{April 02, 2024}
\author{Eduardo Valdivia}

\graphicspath{{./}}
% preamble end --------------------------

% document -----------------
\begin{document}
\maketitle
\section{Design}
The approach when designing both classes was to implement the Alternating-Bit
Protocol, also known as rdt3.0. When implemented right, it would allow for
reliable data transfer that can handle corruption and lost packets. \\

\noindent Some helper methods available to both classes are \texttt{checksum}
and \texttt{trace}. The method \texttt{checksum} takes in ack \#, seq \#, and
message. All it does is add the ack and seq \#, and the message (each byte as an
int). This is used to create a checksum and check if packets are corrupt. The method \texttt{trace} is just a wrapper around \texttt{print} that checks the trace level before printing.

\subsection{Entity A}
This class has many instance variables:
\begin{itemize}
\item \texttt{limit:} holds the upper limit of seqnum
\item \texttt{buffer:} buffer of queued up messages
\item \texttt{current\_ack:} the ack \# expected to receive
\item \texttt{current\_seq:} the seq \# to send
\item \texttt{timeout:} the timeout length
\item \texttt{last\_packet:} the last packet sent
\item \texttt{waiting\_for\_ack:} bool that states if a packet is in transit
\end{itemize}
When a message arrives from layer 5 in \texttt{output()}, it first checks if a
packet is already in transit. If a packet is already in transit add the incoming
message to the buffer. If not, send the next in-order packet by checking if we
have any messages in the queue and start the timer for a timeout. \\

\noindent When a packet arrives from layer 3 in \texttt{input()}, we first check
if the validity of the packet. If the packet is either corrupt or out-of-order,
we will do nothing and return early. If the packet is good, we will stop the
timer, set the state of \texttt{waiting\_for\_ack} to reflect we are not waiting
for an ack anymore, set \texttt{current\_seq} to the next seqnum and if the queue has waiting packets, send the next in-order
packet. \\

\noindent When a timeout occurs, we will resend the last packet sent from \texttt{last\_packet} and restart the timer. \\

\noindent This class also has helper methods \texttt{next\_ack} and
\texttt{send\_from\_buffer}, which helps get the next valid seq/ack \# and send
a message from the buffer.

\subsection{Entity B}
This class has some instance variables:
\begin{itemize}
\item \texttt{limit:} holds the upper limit of seqnum
\item \texttt{current\_ack:} the ack \# to send
\end{itemize}
\noindent When a message arrives from layer 3, it first checks the validity of the
incoming packet. If it is either corrupt or out-of-order, it will simply resend
the previous ack (\texttt{current\_ack - 1}). If the packet is good, we will
send the packet payload out to layer 5, send out an ack, and increment
\texttt{current\_ack}. \\

\noindent This class also has helper methods \texttt{next\_ack}, \texttt{prev\_ack}, and
\texttt{send\_ack}. These methods help get the next and previous ack and send out acks.

\section{Testing}
During testing, running a configuration with low avg. time like: \\\\
\hspace*{10mm} \texttt{python3 rdtsim.py -n 1000 -z 2 -d 10 -l 0.0 -c 0.0 -v 1}
\\\\
\noindent exposed the problems I had with sending packets in-order because I was
forgetting to add the messsage to the queue in \texttt{input()}. \\

\noindent Running a configuration with corruption and loss like: \\\\
\hspace*{10mm} \texttt{python3 rdtsim.py -n 1000 -z 2 -d 100 -l 0.1 -c 0.1 -v 1}
\\\\
\noindent exposed the problems I had with Entity B, not sending the correct ack
\# when a packet was out-of-order or corrupted. This lead to me making the
\texttt{prev\_ack} helper method.
\section{Output}
Output of run to 10 acked packages:

\newcommand{\an}[1]{\hspace*{10mm} \textcolor{red}{#1}}
\texttt{\\1 SIMULATION CONFIGURATION\\
2 --------------------------------------\\
3 (-n) \# layer5 msgs to be provided:      12\\
4 (-d) avg layer5 msg interarrival time:  10.0\\
5 (-z) transport protocol seqnum limit:   2\\
6 (-l) layer3 packet loss prob:           0.05\\
7 (-c) layer3 packet corruption prob:     0.05\\
8 (-s) simulation random seed:            1712085143688875133\\
9 --------------------------------------\\
10\\
11 ===== SIMULATION BEGINS\\
12 A: sent a packet Pkt(seqnum=0, acknum=0, checksum=1940, payload=b'aaaaaaaaaaaaaaaaaaaa')\\
13           \hspace*{10mm} TO\_LAYER3: packet being lost\\
14 A: started timer\\
15 A: waiting for ack 0\\
\an{start of a timer interupt here}\\
16 A: timer interupt!\\
\an{A is resending the packet to fix the interrupt}\\
17 A: resending packet with seqnum 0\\
18 A: starting timer\\
19 A: queued up a message Msg(data=b'bbbbbbbbbbbbbbbbbbbb'), buffer size is 1\\
20 B: sent ack 0\\
21 B: next ack should be 1\\
\an{start of a timer interupt here}\\
22 A: timer interupt!\\
\an{A is resending the packet to fix the interrupt}\\
23 A: resending packet with seqnum 0\\
24 A: starting timer\\
25 A: queued up a message Msg(data=b'cccccccccccccccccccc'), buffer size is 2\\
26 B: got packet seq 0, expected 1\\
27 B: resent  ack 0\\
\an{ack 1}\\
28 A: packet good, stopping timer\\
29 A: sending message from buffer, buffer size is 1\\
30 A: sent a packet Pkt(seqnum=1, acknum=0, checksum=1961, payload=b'bbbbbbbbbbbbbbbbbbbb')\\
31 A: starting timer\\
32 A: waiting for ack 1\\
33 A: queued up a message Msg(data=b'dddddddddddddddddddd'), buffer size is 2\\
34 B: sent ack 1\\
35 B: next ack should be 0\\
36 A: got packet ack 0, expected 1\\
37 A: queued up a message Msg(data=b'eeeeeeeeeeeeeeeeeeee'), buffer size is 3\\
\an{ack 2}\\
38 A: packet good, stopping timer\\
39 A: sending message from buffer, buffer size is 2\\
40 A: sent a packet Pkt(seqnum=0, acknum=0, checksum=1980, payload=b'cccccccccccccccccccc')\\
41 A: starting timer\\
42 A: waiting for ack 0\\
43 B: sent ack 0\\
44 B: next ack should be 1\\
\an{ack 3}\\
45 A: packet good, stopping timer\\
46 A: sending message from buffer, buffer size is 1\\
47 A: sent a packet Pkt(seqnum=1, acknum=0, checksum=2001, payload=b'dddddddddddddddddddd')\\
48 A: starting timer\\
49 A: waiting for ack 1\\
50 A: queued up a message Msg(data=b'ffffffffffffffffffff'), buffer size is 2\\
51 B: sent ack 1\\
52 B: next ack should be 0\\
\an{start of a timer interupt here}\\
53 A: timer interupt!\\
\an{A is resending the packet to fix the interrupt}\\
54 A: resending packet with seqnum 1\\
55 A: starting timer\\
\an{ack 4}\\
56 A: packet good, stopping timer\\
57 A: sending message from buffer, buffer size is 1\\
58 A: sent a packet Pkt(seqnum=0, acknum=0, checksum=2020, payload=b'eeeeeeeeeeeeeeeeeeee')\\
59 A: starting timer\\
60 A: waiting for ack 0\\
61 A: queued up a message Msg(data=b'gggggggggggggggggggg'), buffer size is 2\\
62 B: got packet seq 1, expected 0\\
63 B: resent  ack 1\\
64 B: sent ack 0\\
65 B: next ack should be 1\\
\an{start of a timer interupt here}\\
66 A: timer interupt!\\
\an{A is resending the packet to fix the interrupt}\\
67 A: resending packet with seqnum 0\\
68 A: starting timer\\
69 A: got packet ack 1, expected 0\\
70 B: got packet seq 0, expected 1\\
71 B: resent  ack 0\\
\an{ack 5}\\
72 A: packet good, stopping timer\\
73 A: sending message from buffer, buffer size is 1\\
74 A: sent a packet Pkt(seqnum=1, acknum=0, checksum=2041, payload=b'ffffffffffffffffffff')\\
75 A: starting timer\\
76 A: waiting for ack 1\\
77 A: queued up a message Msg(data=b'hhhhhhhhhhhhhhhhhhhh'), buffer size is 2\\
78 A: got packet ack 0, expected 1\\
79 B: sent ack 1\\
80 B: next ack should be 0\\
\an{ack 6}\\
81 A: packet good, stopping timer\\
82 A: sending message from buffer, buffer size is 1\\
83 A: sent a packet Pkt(seqnum=0, acknum=0, checksum=2060, payload=b'gggggggggggggggggggg')\\
84 A: starting timer\\
85 A: waiting for ack 0\\
86 B: sent ack 0\\
87 B: next ack should be 1\\
\an{ack 7}\\
88 A: packet good, stopping timer\\
89 A: sending message from buffer, buffer size is 0\\
90 A: sent a packet Pkt(seqnum=1, acknum=0, checksum=2081, payload=b'hhhhhhhhhhhhhhhhhhhh')\\
91 A: starting timer\\
92 A: waiting for ack 1\\
93 B: sent ack 1\\
94 B: next ack should be 0\\
95 A: queued up a message Msg(data=b'iiiiiiiiiiiiiiiiiiii'), buffer size is 1\\
\an{ack 8}\\
96 A: packet good, stopping timer\\
97 A: sending message from buffer, buffer size is 0\\
98 A: sent a packet Pkt(seqnum=0, acknum=0, checksum=2100, payload=b'iiiiiiiiiiiiiiiiiiii')\\
\an{packet is going to be corrupted here}\\
99           \hspace*{10mm} TO\_LAYER3: packet being corrupted\\
100 A: starting timer\\
101 A: waiting for ack 0\\
102 A: queued up a message Msg(data=b'jjjjjjjjjjjjjjjjjjjj'), buffer size is 1\\
\an{B finds out packet is bad}\\
103 B: packet checksum is 2100, expected 2085\\
\an{B sends back the previous ack to fix the corruption}\\
104 B: resent  ack 1\\
\an{start of a timer interupt here}\\
105 A: timer interupt!\\
\an{A is resending the packet to fix the interrupt}\\
106 A: resending packet with seqnum 0\\
107 A: starting timer\\
108 A: got packet ack 1, expected 0\\
109 A: queued up a message Msg(data=b'kkkkkkkkkkkkkkkkkkkk'), buffer size is 2\\
110 B: sent ack 0\\
111 B: next ack should be 1\\
\an{start of a timer interupt here}\\
112 A: timer interupt!\\
\an{A is resending the packet to fix the interrupt}\\
113 A: resending packet with seqnum 0\\
114 A: starting timer\\
\an{ack 9}\\
115 A: packet good, stopping timer\\
116 A: sending message from buffer, buffer size is 1\\
117 A: sent a packet Pkt(seqnum=1, acknum=0, checksum=2121, payload=b'jjjjjjjjjjjjjjjjjjjj')\\
\an{packet is going to be lost here}\\
118           \hspace*{10mm} TO\_LAYER3: packet being lost\\
119 A: starting timer\\
120 A: waiting for ack 1\\
121 A: queued up a message Msg(data=b'llllllllllllllllllll'), buffer size is 2\\
122 ===== SIMULATION ENDS\\
123\\
124 SIMULATION SUMMARY\\
125 --------------------------------\\
126 \# layer5 msgs provided to A:      12\\
127 \# elapsed time units:             135.26757669561312\\
128\\
129 \# layer3 packets sent by A:       16\\
130 \# layer3 packets sent by B:       13\\
131 \# layer3 packets lost:            2\\
132 \# layer3 packets corrupted:       1\\
133 \# layer5 msgs delivered by A:     0\\
134 \# layer5 msgs delivered by B:     9\\
135 \# layer5 msgs by B/elapsed time:  0.06653479141015675\\
136 --------------------------------\\
}\\
\subsection{Statistics of run with high loss probability only}
\hspace*{10mm} \texttt{python3 rdtsim.py -n 1000 -z 2 -d 10 -l 0.1 -c 0.0 -v 0}
\texttt{\\\\1 SIMULATION CONFIGURATION\\
2 --------------------------------------\\
3 (-n) \# layer5 msgs to be provided:      1000\\
4 (-d) avg layer5 msg interarrival time:  10.0\\
5 (-z) transport protocol seqnum limit:   2\\
6 (-l) layer3 packet loss prob:           0.1\\
7 (-c) layer3 packet corruption prob:     0.0\\
8 (-s) simulation random seed:            1712087309368015349\\
9 --------------------------------------\\
10 \\
11 SIMULATION SUMMARY\\
12 --------------------------------\\
13 \# layer5 msgs provided to A:      1000\\
14 \# elapsed time units:             9990.264648949254\\
15 \\
16 \# layer3 packets sent by A:       1164\\
17 \# layer3 packets sent by B:       1023\\
18 \# layer3 packets lost:            252\\
19 \# layer3 packets corrupted:       0\\
20 \# layer5 msgs delivered by A:     0\\
21 \# layer5 msgs delivered by B:     698\\
22 \# layer5 msgs by B/elapsed time:  0.06986801896918853\\
23 --------------------------------\\}
\\ This shows that we are delivering more than half of the received messages.
This might look good, but we should be delivering more since layer3 only lost
252 packets. This could be due to a bad timeout value (should probably have it
be a little higher). You can also argue this is to be expected with a low avg. between messages sent (we will see runs with higher averages later).
\\\\
\subsection{Statistics of run with high corruption probability only}
\hspace*{10mm} \texttt{python3 rdtsim.py -n 1000 -z 2 -d 10 -l 0.0 -c 0.1 -v 0}
\texttt{\\1 SIMULATION CONFIGURATION\\
2 --------------------------------------\\
3 (-n) \# layer5 msgs to be provided:      1000\\
4 (-d) avg layer5 msg interarrival time:  10.0\\
5 (-z) transport protocol seqnum limit:   2\\
6 (-l) layer3 packet loss prob:           0.0\\
7 (-c) layer3 packet corruption prob:     0.1\\
8 (-s) simulation random seed:            1712087309449474195\\
9 --------------------------------------\\
10 \\
11 SIMULATION SUMMARY\\
12 --------------------------------\\
13 \# layer5 msgs provided to A:      1000\\
14 \# elapsed time units:             10127.554352250712\\
15 \\
16 \# layer3 packets sent by A:       1187\\
17 \# layer3 packets sent by B:       1187\\
18 \# layer3 packets lost:            0\\
19 \# layer3 packets corrupted:       227\\
20 \# layer5 msgs delivered by A:     0\\
21 \# layer5 msgs delivered by B:     708\\
22 \# layer5 msgs by B/elapsed time:  0.0699082893435824\\
23 --------------------------------\\}
\\ This is a little better than the last one, but still not optimal since we
only lost 227 packets. My guess it's the static timeout value the this solution
has (might get better result with a timeout design like TCP Reno).
\\\\
\subsection{Statistics of run with higher corruption probability and avg. time between msg}
\hspace*{10mm} \texttt{python3 rdtsim.py -n 1000 -z 2 -d 100 -l 0.05 -c 0.1 -v 0}
\texttt{\\\\1 SIMULATION CONFIGURATION\\
2 --------------------------------------\\
3 (-n) \# layer5 msgs to be provided:      1000\\
4 (-d) avg layer5 msg interarrival time:  100.0\\
5 (-z) transport protocol seqnum limit:   2\\
6 (-l) layer3 packet loss prob:           0.05\\
7 (-c) layer3 packet corruption prob:     0.1\\
8 (-s) simulation random seed:            1712088286965552882\\
9 --------------------------------------\\
10\\
11 SIMULATION SUMMARY\\
12 --------------------------------\\
13 \# layer5 msgs provided to A:      1000\\
14 \# elapsed time units:             101570.52695001883\\
15\\
16 \# layer3 packets sent by A:       1639\\
17 \# layer3 packets sent by B:       1549\\
18 \# layer3 packets lost:            161\\
19 \# layer3 packets corrupted:       327\\
20 \# layer5 msgs delivered by A:     0\\
21 \# layer5 msgs delivered by B:     999\\
22 \# layer5 msgs by B/elapsed time:  0.00983553034525056\\
23 --------------------------------\\
}
\\With higher avg. time between messages we recover all packets lost/corrupted. This is to be excepted with rdt3.0.
\\\\
\subsection{Statistics of run with higher loss probability and avg. time between msg}
\hspace*{10mm} \texttt{python3 rdtsim.py -n 1000 -z 2 -d 100 -l 0.1 -c 0.05 -v 0}
\texttt{\\\\1 SIMULATION CONFIGURATION\\
2 --------------------------------------\\
3 (-n) \# layer5 msgs to be provided:      1000\\
4 (-d) avg layer5 msg interarrival time:  100.0\\
5 (-z) transport protocol seqnum limit:   2\\
6 (-l) layer3 packet loss prob:           0.1\\
7 (-c) layer3 packet corruption prob:     0.05\\
8 (-s) simulation random seed:            1712088287055115545\\
9 --------------------------------------\\
10 \\
11 SIMULATION SUMMARY\\
12 --------------------------------\\
13 \# layer5 msgs provided to A:      1000\\
14 \# elapsed time units:             99004.78776932895\\
15 \\
16 \# layer3 packets sent by A:       1598\\
17 \# layer3 packets sent by B:       1460\\
18 \# layer3 packets lost:            267\\
19 \# layer3 packets corrupted:       140\\
20 \# layer5 msgs delivered by A:     0\\
21 \# layer5 msgs delivered by B:     999\\
22 \# layer5 msgs by B/elapsed time:  0.010090421104962803\\
23 --------------------------------\\
}
\\Same as the last one we recover all packets lost/corrupted. Might be safe to assume that a good timeout value is important.

\end{document}
